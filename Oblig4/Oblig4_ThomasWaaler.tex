\documentclass[10pt,a4paper]{article}

\usepackage[utf8]{inputenc}
\usepackage{biblatex}
\usepackage{amsmath}
\usepackage{amsfonts}
\usepackage{amssymb}
\usepackage{enumitem}
\usepackage{xcolor}
\usepackage{listings}

\author{Thomas Waaler}
\title{Obligatorisk Oppgave 4}

\definecolor{codegreen}{rgb}{0,0.6,0}
\definecolor{codegray}{rgb}{0.5,0.5,0.5}
\definecolor{codepurple}{rgb}{0.58,0,0.82}
\definecolor{backcolour}{rgb}{0.95,0.95,0.92}

\lstdefinestyle{mystyle}{
    backgroundcolor=\color{backcolour},   
    commentstyle=\color{codegreen},
    keywordstyle=\color{magenta},
    numberstyle=\tiny\color{codegray},
    stringstyle=\color{codepurple},
    basicstyle=\ttfamily\footnotesize,
    breakatwhitespace=false,         
    breaklines=true,                 
    captionpos=b,                    
    keepspaces=true,                 
    numbers=left,                    
    numbersep=5pt,                  
    showspaces=false,                
    showstringspaces=false,
    showtabs=false,                  
    tabsize=2
}

\lstset{style=mystyle}

\begin{document}
\maketitle


\section*{Teorioppgave 1 - Exception}
Når en error oppstår blir en Exception objekt hevet/raised for å signalisere hva slags error type som oppstår. Dersom den exception ikke blir behandlet vil den stoppe programmet og printe ut error meldingen. Får å håndtere en exception selv kan man putte metoden som skaper denne exception inn i en try except.

Et eksempel hvor det er relevant å bruke dette er når man tar input fra brukeren og skal omgjøre dette til en int men brukeren spesifiserer ikke en int. Da vil ValueError exception bli hevet/raised og man kan håndtere dette.

\begin{lstlisting}[language=Python]
	while True:
		try:
			user_input = int( input("Enter a whole number: ") )
			break	# Value converted, break out of loop
		except ValueError:
			print("[ERROR]: Please insert ONLY a WHOLE number")
\end{lstlisting}

\section*{Teorioppgave 2 - Klasse}
En klasse er en gruppering av variabler og metoder spesifikt til den klassen.

\begin{lstlisting}[language=Python]
	class Item:
		def __init__(self, name: str, type: str, count: int=1):
			# Variables related to the class			
			self.name = name
			self.type = type			
			self.count = count
		
		def __str__(self):
			'''Returned when turning object into a string'''
			return f"Item '{self.name.title()}' of type {self.type} with a count of {self.count}"
		
		# Method owned by the class
		def increase_count(self, num: int=1):
			self.count += num
\end{lstlisting}

\section*{Teorioppgave 3 - Objekt}
Et objekt hva man får når man initialiserer en klasse. Som vil si at en klasse er konstruktsjonen av et objekt. For å lage et objekt kan man gjøre følgene:


\begin{lstlisting}[language=Python]
	# Using class from last example
	
	my_item = Item("Bread", "Food", 1)	# Initializing the object
	print(my_item)
	my_item.increase_count(2)			# Calling a method in the object
\end{lstlisting}


\end{document}
