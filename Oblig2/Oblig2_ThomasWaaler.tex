\documentclass[10pt,a4paper]{article}

\usepackage[utf8]{inputenc}
\usepackage{biblatex}
\usepackage{amsmath}
\usepackage{amsfonts}
\usepackage{amssymb}
\usepackage{xcolor}
\usepackage{listings}

\author{Thomas Waaler}
\title{Obligatorisk Oppgave 2}

\definecolor{codegreen}{rgb}{0,0.6,0}
\definecolor{codegray}{rgb}{0.5,0.5,0.5}
\definecolor{codepurple}{rgb}{0.58,0,0.82}
\definecolor{backcolour}{rgb}{0.95,0.95,0.92}

\lstdefinestyle{mystyle}{
    backgroundcolor=\color{backcolour},   
    commentstyle=\color{codegreen},
    keywordstyle=\color{magenta},
    numberstyle=\tiny\color{codegray},
    stringstyle=\color{codepurple},
    basicstyle=\ttfamily\footnotesize,
    breakatwhitespace=false,         
    breaklines=true,                 
    captionpos=b,                    
    keepspaces=true,                 
    numbers=left,                    
    numbersep=5pt,                  
    showspaces=false,                
    showstringspaces=false,
    showtabs=false,                  
    tabsize=2
}

\lstset{style=mystyle}

\begin{document}
\maketitle


\section*{Teorioppgave 1 - if-test}
En if-test er en måte å sjekke om ett eller flere uttrykk er sant eller usant.
Dersom man ønsker å sjekke flere forskjellige uttrykk etter hverandre men ønsker at bare én av if-blokkene skal bli kjørt så kan man bruke \textcolor{orange}{\textbf{if}}, \textcolor{orange}{\textbf{elif}} og \textcolor{orange}{\textbf{else}}. For eksempel:

\begin{lstlisting}[language=Python]
variable = 20
if variable <= 0:
	# Enters this codeblock only when "variable" is less than or
	# equal to 0
elif variable == 22:
	# Enters this codeblock only when "variable" is equal to 22
else:
	# Enters this codeblock only when the two other if statements
	# haven't been hit
\end{lstlisting}


\section*{Teorioppgave 2 - for-løkke}
En for-løkke er en metode å iterere \textbf{x} antall ganger som oppgitt, f.eks. gjennom uttrykk, og utførerer en kode-blokk for hver iterasjon.

\subsection*{Eksempel 1 - Løkke med \textcolor{violet}{\textbf{range}}()}
Si vi har en kode-blokk som vi har lyst til å kjøre flere ganger. Istedenfor å skrive den samme kode-blokken flere ganger så kan man benytte en for-løkke. Som vist:
\begin{lstlisting}[language=Python]
# We want to run a code-block 6 times
for numb in range(7):	# Range from 0 to 6 = 7
	random_var = numb ** 2
	other_var = (random_var + numb) * 0.25
	
	DistantFunction(other_var)
\end{lstlisting}

\subsection*{Eksempel 2 - Løkke på \textcolor{teal}{List}}
Si vi har en liste over spill hvor vi ønsker å printe ut hvert element med litt tekst. Da kan man bruke en for-løkke på listen. Som vist:
\begin{lstlisting}[language=Python]
# One way of doing this
games_list = ['Deep Rock Galactic', 'World of Warcraft', 
			  'Guild Wars 2', 'Dark Souls 3', 'Code Vein']

# Current element is passed into 'game' variable
for game in games_list: 
	print(f"{game} is fun to play!")
\end{lstlisting}


\section*{Teorioppgave 3 - liste}
En \textcolor{teal}{List} er et variabel som inneholder elementer med data (f.eks \textcolor{teal}{String}, \textcolor{teal}{Int}, \textcolor{teal}{Float}, osv). \textcolor{teal}{List} kan brukes til å lagre en kolleksjon av data, for eksempel en liste over planeter i solsystemet vårt, sånn at man slipper å lage ett variabel for hver planet. Som vist:
\begin{lstlisting}[language=Python]
# Instead of doing this
mercury_name = "Mercury"
venus_name = "Venus"
earth_name = "Earth"

# We can do this
planet_names = ["Mercury", "Venus", "Earth"]
\end{lstlisting}

\end{document}
